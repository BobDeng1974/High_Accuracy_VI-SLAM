\chapter{Introduction}
\label{sec:introduction}
%\chapter{Einleitung}
%\label{sec:einleitung}


%State estimation is a fundamental problem in a wide range of applications, 
%including robotic navigation, autonomous driving, virtual and augmented 
%reality. 



\acf{SLAM} is the task of navigating in a previously unknown environment by 
building a map of the environment which the robot is in and simultaneously 
estimating its position within this map. \\

\ac{SLAM} systems can usually be divided into two basic components, the 
front-end and the back-end. The front-end builds a local map and estimates the 
robot's pose incrementally within this local map. Small deviations are 
accumulated over time which results in a drift of the estimated pose. If a 
previously visited location is detected, the back-end can 
correct for the drift with the help of loop closures and pose graph 
optimization. 
Although, a globally topologically consistent map can be achieved this way, the 
accumulated 
error cannot be removed completely. This means, the accuracy of the front-end 
is fundamental for the overall performance of a \ac{SLAM} system. \\

Cameras are now easily found in many consumer electronic products. This makes 
systems using a single camera very appealing due to their small size and low 
power consumption. Furthermore, cameras can operate under different light 
conditions, both indoors and outdoors. Hence, monocular front-end systems 
have become very popular in recent years. However, it is known that a visual 
only front-end, implemented as monocular \ac{VO} system, cannot retrieve the 
scale of a scene. A popular solution for this problem is the integration of an 
\ac{IMU}. The rich representation of a scene captured with a camera, together 
with the accurate short-term movement estimates by the \ac{IMU} have been 
acknowledged to complement each other well.\\ 

The performance of \ac{VIO} system is often evaluated using \acp{UAV}. Due to 
their fast dynamics and 6 \ac{DoF} movements, they represent the most 
challenging type of robot. However, \acp{UAV} normally have high power and 
payload constraints and the state estimation must work on embedded hardware 
with limited computational resources. \\ 

Monocular \ac{SLAM} or \ac{VIO} solutions are either filter-based (e.g. using an
\ac{EKF}) or optimization based using keyframes. It has been shown 
\citep{Strasdat2010} that keyframe based methods outperform filter-based ones, 
given enough computational power. Hence, most new releases of monocular 
\ac{SLAM} systems are keyframe based. Keyframes provide the additional 
advantage that they can be used by simultaneously running 
algorithms at the same time. Keyframes can for example be exchanged between two 
robots to perform collaborative \ac{SLAM} \citep{schmuck2017}, 
\citep{Karrer2018} or keyframes can be used for path planning tasks 
\citep{alzugaray2017short}. Running other tasks simultaneously beside the 
\ac{SLAM} system increases the computational payload significantly, reducing the 
available computational resources dedicated for the \ac{SLAM} task.\\

This work proposes an accurate keyframe based monocular \ac{VIO} system for 
computationally restricted platforms. The systems builds on an existing 
open-source \ac{VIO} implementation VINS-Mono \citep{Qin2017VINS}. The 
impact of the different parameter settings in the initial implementation of 
VINS-Mono are analyzed on a UP Board, a single board computer. 
Based on this evaluation an adaption is proposed in order to achieve an 
implementation with real-time performance. The resulting system is extensively
tested on all 
sequences of the EuRoC micro aerial vehicle dataset \citep{Burri2016EuRoC}. 
Furthermore, the system is deployed on a real \ac{UAV} to ensure its 
functionality. 

